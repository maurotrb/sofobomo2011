%
% Senigallia Cemetery. The Old Part.
% (c) 2011 Mauro Taraborelli (MauroTaraborelliPhoto@gmail.com).
% This work is licensed under a Creative Commons Attribution-Share Alike
% 3.0 Unported License
% See http://creativecommons.org/licenses/by-sa/3.0/
%
% Mauro Taraborelli makes no representation about the suitability or accuracy
% of this software or data for any purpose, and makes no warranties,
% either express or implied, including merchantability and fitness for a
% particular purpose or that the use of this software or data will not
% infringe any third party patents, copyrights, trademarks, or other
% rights.  The software and data are provided "as is". 
% 
% Derived from the work 'Chickens, Anyone?' by Eric R. Jeschke
% http://redskiesatnight.com/books/pod/latex-templates-for-pod-publishing-with-blurb-com/
% (c) 2006 Eric R. Jeschke (eric@redskieatnight.com).
%
% [top-level file]
\nonstopmode

\documentclass
[ 10pt ,
  final ,
  openany ]
{book}

% Used to set page size and margin
% Same dimensions as Blurb book, without the bindingoffset
\usepackage
[ paperwidth=9.625in ,
  paperheight=8.25in , 
  % amount we 'lose' (visually) due to the binding
  % left and right pages are offset a bit to the sides
  % (typically looks better for most kinds of bindings)
  bindingoffset=0.25in,
  % set total to the dimensions of the printed part
  total={8in,7.75in} ,
  % include header/footer/margin notes in printed area
  twoside ,
  includeall ,
  nomarginpar ,
  ignorehead=false ,
  ignorefoot=false ,
  ignoremp=false ,
  % center printed area on page
  vcentering ,
  hcentering ]
{geometry}
% width of margin notes area
%\setlength{\marginparwidth}{0in}
% distance between margin and paragraph
%\setlength{\marginparsep}{0in}
% footer configuration 
\setlength{\footnotesep}{10pt}
% header configuration 
\setlength{\headheight}{10pt}
\setlength{\headsep}{0.1in}
% paragraph indentations
\setlength{\parindent}{0in}
% amount of space before each new paragraph begins
\setlength{\parskip}{1em}

% used to include the photos
\usepackage{graphicx}
% where can I find the photos that will be imported for this version
\graphicspath{{./photos.book/}}

% used for QR code
\usepackage{pstricks}
\usepackage{pst-barcode}

% used for multilingual text
\usepackage{polyglossia}
\setdefaultlanguage[variant=american]{english}
\setotherlanguages{italian}
% used for parallel columns
\usepackage[SeparatedFootnotes]{parallel}

% in case you want to embed any hyperlinks
% turn on colorlinks to avoid the nasty box around the link
\usepackage
[ colorlinks=true ,
  urlcolor=black ]
{hyperref}

% used during design, to create lore ipsum text
\usepackage{lipsum}

% Metadata
\title{Senigallia Cemetery. The Old Part.}
\author{Mauro Taraborelli}
%%%%%%%%% PDF stuff, IF USING xelatex %%%%%%%%%
\special{pdf:docinfo <<
/Title (Senigallia Cemetery. The Old Part.) % set your title here
/Author (Mauro Taraborelli) % set author name
/Subject (Senigallia Cemetery) % set subject
/Keywords (cemetery, colombarium, gates, flowers, decorations, Italy, photography) % set keywords
/Trapped (False)
/GTS_PDFXVersion (PDF/X-3:2002)
% must have a trim box, but I think Blurb ignores the values
/TrimBox [0.00000 9.00000 684.36000 585.00000] >>
}
\special{pdf:put @catalog <<
/OutputIntents [ <<
/Info (none)
/Type /OutputIntent
/S /GTS_PDFX
/OutputConditionIdentifier (Blurb.com)
/RegistryName (http://www.color.org/)
>> ] >>
}

% include common fonts definition
%
% Senigallia Cemetery. The Old Part.
% (c) 2011 Mauro Taraborelli (MauroTaraborelliPhoto@gmail.com).
% This work is licensed under a Creative Commons Attribution-Share Alike
% 3.0 Unported License
% See http://creativecommons.org/licenses/by-sa/3.0/
%
% Mauro Taraborelli makes no representation about the suitability or accuracy
% of this software or data for any purpose, and makes no warranties,
% either express or implied, including merchantability and fitness for a
% particular purpose or that the use of this software or data will not
% infringe any third party patents, copyrights, trademarks, or other
% rights.  The software and data are provided "as is". 
% 
% Derived from the work 'Chickens, Anyone?' by Eric R. Jeschke
% http://redskiesatnight.com/books/pod/latex-templates-for-pod-publishing-with-blurb-com/
% (c) 2006 Eric R. Jeschke (eric@redskieatnight.com).
%
% [common fonts definitions]
%
% You might be able to use opentype tools to find out the names of
% the fonts you can use here, or you can just experiment
%  $ otfinfo -a .../*.otf
%
\usepackage{fontspec}% font selecting commands
\usepackage{xunicode}% unicode character macros
\usepackage{xltxtra} % a few fixes and extras

\defaultfontfeatures{Mapping=tex-text,Scale=MatchLowercase}     % to support TeX conventions like ‘‘--
\setmainfont
[ UprightFont = {* Regular} ,
  BoldFont = {* Bold} ,
  ItalicFont = {* Italic} ,
  SmallCapsFont = {Fontin SmallCaps} ] % Can't use * for Fontin SmallCaps. See http://web.archiveorange.com/archive/v/tMRBZ9YChLvsIQr9Lqh2
{Fontin}
\setsansfont
[ UprightFont = {* Regular} ,
  BoldFont = {* Bold} ,
  ItalicFont = {* Italic} ,
  SmallCapsFont = {Fontin Sans Small Caps} ]
{Fontin Sans}
\setmonofont
[ UprightFont = {*} ,
  BoldFont = {* Bold} ,
  ItalicFont = {* Italic} ]
{Lekton}

% include common color definition
%
% Senigallia Cemetery. The Old Part.
% (c) 2011 Mauro Taraborelli (MauroTaraborelliPhoto@gmail.com).
% This work is licensed under a Creative Commons Attribution-Share Alike
% 3.0 Unported License
% See http://creativecommons.org/licenses/by-sa/3.0/
%
% Mauro Taraborelli makes no representation about the suitability or accuracy
% of this software or data for any purpose, and makes no warranties,
% either express or implied, including merchantability and fitness for a
% particular purpose or that the use of this software or data will not
% infringe any third party patents, copyrights, trademarks, or other
% rights.  The software and data are provided "as is". 
% 
% Derived from the work 'Chickens, Anyone?' by Eric R. Jeschke
% http://redskiesatnight.com/books/pod/latex-templates-for-pod-publishing-with-blurb-com/
% (c) 2006 Eric R. Jeschke (eric@redskieatnight.com).
%
% [common color definitions]

% Old Gold
\definecolor{MyGold}{rgb}{0.81,0.71,0.23}

% Medium Gray
\definecolor{MyGray}{rgb}{0.38,0.38,0.38}


% begin the document and suppress page numbers
\begin{document}
\pagestyle{empty}

% interior facing page for book version
\begin{center}

\vspace*{2in}

{\fontsize{.4in}{.4in}\selectfont \textbf{\textsf{\textsc{Mauro Taraborelli}}}}

\vspace*{0.5in}

{\fontsize{.6in}{.6in}\selectfont \textbf{\textsf{Senigallia Cemetery. The Old Part.}}}

{\fontsize{.4in}{.4in}\selectfont \textbf{\textsf{Il cimitero di Senigallia. La parte antica.}}}

\end{center}

\newpage

% bring in the rest of the content which is in common with the book version
%
% Senigallia Cemetery. The Old Part.
% (c) 2011 Mauro Taraborelli (MauroTaraborelliPhoto@gmail.com).
% This work is licensed under a Creative Commons Attribution-Share Alike
% 3.0 Unported License
% See http://creativecommons.org/licenses/by-sa/3.0/
%
% Mauro Taraborelli makes no representation about the suitability or accuracy
% of this software or data for any purpose, and makes no warranties,
% either express or implied, including merchantability and fitness for a
% particular purpose or that the use of this software or data will not
% infringe any third party patents, copyrights, trademarks, or other
% rights.  The software and data are provided "as is". 
% 
% Derived from the work 'Chickens, Anyone?' by Eric R. Jeschke
% http://redskiesatnight.com/books/pod/latex-templates-for-pod-publishing-with-blurb-com/
% (c) 2006 Eric R. Jeschke (eric@redskieatnight.com).
%
% common information to the web version and book version
%
% images will be imported by searching the paths set with \graphicspath
% the book.tex or web.tex sets this so the correct resolution images end
% up in the correct document.
% 
% cover is done elsewhere, as this is usually broken out from the 
% ``text block'' for POD publishing--see book.tex
%

\vspace*{\fill}

{\large Copyright \textcopyright{} 2011 Mauro Taraborelli.}

\begin{Parallel}{0.47\textwidth}{0.47\textwidth}
\ParallelLText{
    \begin{english}[variant=american]
        {\large All Rights Reserved.}

        This book may not be reproduced in any form without written permission of the author. 
    \end{english}
}
\ParallelRText{
    \begin{italian}
        {\large Tutti i diritti riservati.}

        Questo libro non può essere riprodotto in alcuna forma senza il permesso scritto dell'autore.
    \end{italian}
}
\ParallelPar
\ParallelLText{
    \begin{english}[variant=american]
        First Edition. PDF created \today.
    \end{english}
}
\ParallelRText{
    \begin{italian}
        Prima edizione. PDF creato il \today.
    \end{italian}
}
\ParallelPar
\end{Parallel}

\begin{center}
    {\tt MauroTaraborelliPhoto@gmail.com}

    {\tt www.maurotaraborelliphoto.com}
\end{center}

\newpage

\pagecolor{MyGray}
\vspace*{2in}

\hfill{\textsf{\textcolor{white}{\Huge Preface}}}

\hfill{\textsf{\textcolor{white}{\LARGE Prefazione}}}

\clearpage{\pagestyle{empty}\cleardoublepage}

\pagecolor{white}
\pagestyle{plain}

\begin{Parallel}{0.47\textwidth}{0.47\textwidth}
\ParallelLText{
    \begin{english}[variant=american]
        Senigallia is a port town on Italy's Adriatic coast, in the Marche region.
        The town is one of the most famous seaside resorts of the region and in
        the summer months its beaches are full of tourists.
    \end{english}
}
\ParallelRText{
    \begin{italian}
        Senigallia è una città di mare sulla costa adriatica italiana, nella regione Marche.
        La città è una delle più famose località turistiche della regione e, nei mesi estivi,
        le sue spiagge sono piene di turisti.
    \end{italian}
}
\ParallelPar
\ParallelLText{
    \begin{english}[variant=american]
        Imagine for a moment that you are on vacation in Senigallia.
        Leave for a few hours the crowded beach and drive your car to the hills that surround Senigallia,
        directed to the town of Corinaldo.
        After a very few miles turn right, leaving the main road, and reach the top of a small hill.
        You will see a little square with a lawn surrounded by trees and an ancient church:
        that is \emph{Santa Maria delle Grazie}.
        Leave your car and go toward the church.
        On the left you will see a simple gate: it will be open.
        Walk past the gate and follow the road around the church.
        You are walking in a little wood called \emph{Selva delle Grazie}.
        After a hundred and fifty feet you will walk past another gate:
        now you are in the Senigallia cemetery.
    \end{english}
}
\ParallelRText{
    \begin{italian}
        Immagina per un momento di essere in vacanza a Senigallia.
        Lascia per poche ore la spiaggia affollata e vai con la tua auto verso le colline
        che circondano Senigallia, diretto alla città di Corinaldo.
        Dopo pochissimi chilometri gira a destra, lasciando la strada principale,
        e raggiungi la cima di una piccola collina.
        Vedrai una piccola piazza con un prato circondato da alberi e una chiesa antica:
        si tratta di Santa Maria delle Grazie.
        Lascia l'auto e vai verso la chiesa. Sulla sinistra vedrai una semplice cancello:
        sarà aperto.
        Supera il cancello e segui la strada che costeggia la chiesa.
        Stai camminando in un piccolo bosco chiamato la \emph{Selva delle Grazie}.
        Dopo cinquanta metri supererai un altro cancello:
        adesso sei nel cimitero di Senigallia.
    \end{italian}
}
\ParallelPar
\ParallelLText{
    \begin{english}[variant=american]
        Senigallia cemetery is an example of 19\textsuperscript{th} century monumental cemetery
        such as those of Milan, Turin, Naples
        \footnote{In 1804, the Napoleon's Edict of Saint--Cloud forbade burial 
        in churches and within settlements for hygienic reasons and foresaw the compulsory
        building of cemeteries outside the populated areas.
        In 1806 the edict was applied to the Kingdom of Italy and in 1809 to the Papal State,
        where Senigallia was located at the time.
        With the new regulations in place, a Renaissance in cemetery planning and architecture
        took place. Many of the monumental cemetery of the big Italian cities dates back to
        this period. But outside the few rich Italian cities the application of the edict was
        much slower.}.
        It is less rich and more chaotic than those cemeteries, but interesting nonetheless,
        if you consider its troubled history and the meager means that a small town like
        Senigallia had to build it
        \footnote{Senigallia tried to build a cemetery suitable for its population during
        all the 19\textsuperscript{th} century. But economic quandary and difficulty to find
        a proper location slowed down the decision.
        The definitive site was decided in the 1869 and the cemetery opened in 1871.
        But the original project was not completed until the '20 of the 20\textsuperscript{th} century.
        See the recommended reading at the end of the book.}.
    \end{english}
}
\ParallelRText{
    \begin{italian}
        Il cimitero di Senigallia è un esempio di cimitero monumentale ottocentesco come quelli
        di Milano, Torino, Napoli.
        \footnote{Nel 1804, l'editto napoleonico di Saint--Cloud vietò la sepoltura
        nelle chiese e all'interno di luoghi abitati per ragioni igieniche e rese obbligatorio
        la costruzione di cimiteri al di fuori delle aree popolate.
        Nel 1806 l'editto fu applicato al Regno di Italia e nel 1809 allo Stato Papale,
        dove Senigallia era presente all'epoca.
        Con la nuova legge, si avviò un Rinascimento nella progettazione e nell'architettura dei cimiteri.
        Molti dei cimiteri monumentali delle grandi città italiane risalgono a questo periodo.
        Ma al di fuori delle poche città italiane ricche, l'applicazione dell'editto fu molto più lenta.}
        È meno ricco e più caotico di quei cimiteri, ma è comunque interessante,
        se tieni conto della sua storia travagliata e degli scarsi mezzi per costruirlo che aveva
        una piccola città come Senigallia.
        \footnote{Senigallia tentò di costruire un cimitero adatto alla sua popolazione nel corso
        di tutto il XIX secolo.
        Ma problemi economici e difficoltà nel trovare un luogo adatto rallentarono la decisione.
        Il sito definitivo fu deciso nel 1869 ed il cimitero aprì nel 1871.
        Ma il progetto originario non fu completato che negli anni 20 del XX secolo.
        Vedi le letture consigliate alla fine del libro.}
    \end{italian}
}
\ParallelPar
\ParallelLText{
    \begin{english}[variant=american]
        The cemetery covers the whole hill beneath you, but the actual size is hidden behind trees and tombs.
        You will see the oldest part of the cemetery, with terraces on the hillside slopes
        divided by a grid of small roads.
        The original layout was of rectangular fields with graves at the center and monumental tombs on the borders.
        As times went by, lack of burial space forced to fill the fields with individual or family tombs.
        Moreover some of the older tombs were demolished and rebuilt.
        So you will see ancient and modern tombs mixed together.
    \end{english}
}
\ParallelRText{
    \begin{italian}
        Il cimitero copre l'intera collina sotto di te, ma la reale estensione è nascosta
        dagli alberi e dalle tombe.
        Vedrai la parte più antica del cimitero, con terrazzamenti lungo la collina divisi
        da una griglia di piccole strade.
        La disposizione iniziale prevedeva campi comuni rettangolari circondati da sepolcri monumentali.
        Col passare del tempo, la mancanza di spazio per le nuove sepolture ha costretto a
        riempire i campi comuni con sepolcri individuali o di famiglia.
        Inoltre alcune delle tombe più antiche sono state demolite e ricostruite.
        Così vedrai tombe antiche e tombe moderne insieme.
    \end{italian}
}
\ParallelPar
\ParallelLText{
    \begin{english}[variant=american]
        If you come here in the early morning or at noon, you will probably be the only visitor.
        Take your time, wander through its rows walking slowly and looking at the details.
        You will discover its visual interest, its historic interest, its human interest.        
    \end{english}
}
\ParallelRText{
    \begin{italian}
        Se vieni qui di primo mattino o a mezzogiorno, sarai probabilmente il solo visitatore.
        Prenditi tempo, vaga tra le sue file camminando lentamente e facendo attenzione ai dettagli.
        Scoprirai il suo interesse visuale, il suo interesse storico, il suo interesse umano.
    \end{italian}
}
\ParallelPar
\ParallelLText{
    \begin{english}[variant=american]
        I myself wandered through the old part of the Senigallia cemetery in a hot summer day.
        The photos in this book gives a glimpse of what I saw and of what you could see.
        The themes in which I divided the photos are a suggestion for you to start your
        exploration. They certainly don't exhaust the visual opportunities of the cemetery. 
    \end{english}
}
\ParallelRText{
    \begin{italian}
        Ho vagato io stesso nella parte antica del cimitero di Senigallia in una calda giornata estiva.
        Le foto di questo libro sono un accenno di cosa ho visto e di cosa tu puoi vedere.
        I temi in cui ho diviso le foto sono un suggerimento per iniziare la tua esplorazione.
        Non esauriscono certo le possibilità visuali del cimitero.
    \end{italian}
}
\ParallelPar
\ParallelLText{
    \begin{english}[variant=american]
        \textbf{Colombarium}
        \footnote{
            A columbarium is a place for the respectful and usually public storage of cinerary urns
            (i.e. urns holding a deceased’s cremated remains).
            The term comes from the Latin columba (dove) and originally referred to compartmentalized
            housing for doves and pigeons (see dovecote). See more details at
            \href{http://en.wikipedia.org/wiki/Columbarium}{en.wikipedia.org/wiki/Columbarium}.
        }.
        The colombarium located in this part of the cemetery dates back to the the first years
        of the 20\textsuperscript{th} century, and it shows.
        The conditions are poor and many niches are no more maintained.
        But you will find images that takes you back to another era and some clues that somebody
        still care about these old tombs. 
    \end{english}
}
\ParallelRText{
    \begin{italian}
        \textbf{Colombari}
        \footnote{
            Il Colombario (dal latino columbarium) è un tipo di costruzione funeraria divisa in loculi
            orizzontali ciascuno dei quali atto ad ospitare una bara. In archeologia per colombario
            si intende un tipo di camera sepolcrale composta da nicchie in cui venivano conservate le urne
            con le ceneri dei defunti. Il nome deriva dal fatto che le nicchie erano ricavate nella muratura
            con apertura anteriore che ricorda appunto le costruzioni per il ricovero e l'allevamento dei colombi.
            Per maggiori dettagli vedi
            \href{http://it.wikipedia.org/wiki/Colombario}{it.wikipedia.org/wiki/Colombario}.
        }.
        Il colombario presente in questa parte del cimitero risale ai primi anni del XX secolo, e si vede.
        Le condizioni sono pessime e molte nicchie non sono più manutenute.
        Ma troverai immagini che ti porteranno indietro ad un altra era e qualche indizio che
        qualcuno si preoccupa ancora di queste vecchie tombe.
    \end{italian}
}
\ParallelPar
\ParallelLText{
    \begin{english}[variant=american]
        \textbf{Gates}.
        Mausoleums act as both monument and place of interment, usually for well-off families.
        I was fascinated by their gates, especially by those of the oldest mausoleums:
        their forms, their styles, their conditions.
    \end{english}
}
\ParallelRText{
    \begin{italian}
        \textbf{Cancelli}.
        Le cappelle sono sia dei monumenti che dei luoghi di sepoltura, di solito per le famiglie abbienti.
        Sono stato affascinato dai loro cancelli, specialmente da quelli delle cappelle più antiche:
        la loro forma, il loro stile, la loro condizione.
    \end{italian}
}
\ParallelPar
\ParallelLText{
    \begin{english}[variant=american]
        \textbf{Flowers and Plants}.
        Flowers and plants are used to grace the tomb and honor the loved ones.
        But sometimes they are forgotten and parched by the time and by the heat.
    \end{english}
}
\ParallelRText{
    \begin{italian}
        \textbf{Fiori e piante}.
        Fiori e piante sono usate per decorare le tombe e onorare le persone amate.
        Ma qualche volta sono dimenticati e seccati dal tempo e dal caldo.
    \end{italian}
}
\ParallelPar
\ParallelLText{
    \begin{english}[variant=american]
        \textbf{Decorations}.
        Simple as a crying woman. Complex and with many symbols.
        One of the most interesting aspects of Senigallia cemetery.
        See the recommended reading at the end of the book for more details.
    \end{english}
}
\ParallelRText{
    \begin{italian}
        \textbf{Decorazioni}.
        Semplici come una donna che piange. Complesse per i molti simboli.
        Uno degli aspetti più interessanti del cimitero di Senigallia.
        Vedi le letture consigliate alla fine del libro per approfondire.
    \end{italian}
}
\ParallelPar
\ParallelLText{
    \begin{english}[variant=american]
    \end{english}
}
\ParallelRText{
    \begin{italian}
    \end{italian}
}
\ParallelPar
\end{Parallel}

\newpage

%To center all the caption
\begin{center}

\pagecolor{MyGray}
\pagestyle{empty}
\vspace*{2in}

\hfill{\textsf{\textcolor{white}{\Huge Senigallia Cemetery. The Old Part.}}}

\hfill{\textsf{\textcolor{white}{\LARGE Il cimitero di Senigallia. La parte antica.}}}

\clearpage{\pagestyle{empty}\cleardoublepage}

\pagecolor{white}
\pagestyle{plain}

\includegraphics[width=8in]{sofobomo2011-1}

{Cemetery Entrance, from \emph{Selva delle Grazie}}\\
{\footnotesize Entrata del cimitero, dalla Selva delle Grazie}
\vspace*{\fill}
\newpage

\includegraphics[width=8in]{sofobomo2011-2}

{The Battaglia Family Grave, Detail}\\
{\footnotesize La tomba della famiglia Battaglia, dettaglio}
\vspace*{\fill}
\newpage

\pagestyle{empty}
\vspace*{2in}

\hfill{\textsf{\Huge Colombarium}}

\hfill{\textsf{\LARGE Colombari}}

\clearpage{\pagestyle{empty}\cleardoublepage}

\pagestyle{plain}

\includegraphics[width=8in]{sofobomo2011-3}

{Colombarium behind Mausoleums Grove, Detail}\\
{\footnotesize Colombario dietro il boschetto delle cappelle, dettaglio}
\vspace*{\fill}
\newpage

\includegraphics[width=8in]{sofobomo2011-4}

{Colombarium behind Mausoleums Grove, Detail}\\
{\footnotesize Colombario dietro il boschetto delle cappelle, dettaglio}
\vspace*{\fill}
\newpage

\includegraphics[width=8in]{sofobomo2011-5}

{Colombarium behind Mausoleums Grove, Detail}\\
{\footnotesize Colombario dietro il boschetto delle cappelle, dettaglio}
\vspace*{\fill}
\newpage

\includegraphics[width=8in]{sofobomo2011-6}

{Liberty Colombarium, Detail}\\
{\footnotesize Colombario liberty, dettaglio}
\vspace*{\fill}
\newpage

\includegraphics[width=8in]{sofobomo2011-7}

{Liberty Colombarium, Detail}\\
{\footnotesize Colombario liberty, dettaglio}
\vspace*{\fill}
\newpage

\includegraphics[width=8in]{sofobomo2011-8}

{Liberty Colombarium, Detail}\\
{\footnotesize Colombario liberty, dettaglio}
\vspace*{\fill}
\newpage

\includegraphics[width=8in]{sofobomo2011-9}

{Liberty Colombarium, Detail}\\
{\footnotesize Colombario liberty, dettaglio}
\vspace*{\fill}
\newpage

\includegraphics[width=8in]{sofobomo2011-10}

{Liberty Colombarium, Detail}\\
{\footnotesize Colombario liberty, dettaglio}
\vspace*{\fill}
\newpage

\pagestyle{empty}
\vspace*{2in}

\hfill{\textsf{\Huge Gates}}

\hfill{\textsf{\LARGE Cancelli}}

\clearpage{\pagestyle{empty}\cleardoublepage}

\pagestyle{plain}

\includegraphics[width=8in]{sofobomo2011-11}

{The Angeloni Family Mausoleum Gate}\\
{\footnotesize Cancello della cappella della famiglia Angeloni}
\vspace*{\fill}
\newpage

\includegraphics[width=8in]{sofobomo2011-12}

{The Adami--Manara Family Mausoleum Gate}\\
{\footnotesize Cancello della cappella della famiglia Adami--Manara}
\vspace*{\fill}
\newpage

\includegraphics[width=8in]{sofobomo2011-13}

{The Clari Family Mausoleum Gate}\\
{\footnotesize Cancello della cappella della famiglia Clari}
\vspace*{\fill}
\newpage

\includegraphics[width=8in]{sofobomo2011-14}

{The Clari Family Mausoleum Gate}\\
{\footnotesize Cancello della cappella della famiglia Clari}
\vspace*{\fill}
\newpage

\includegraphics[width=8in]{sofobomo2011-15}

{The Girolimini Family Mausoleum Gate}\\
{\footnotesize Cancello della cappella della famiglia Girolimini}
\vspace*{\fill}
\newpage

\includegraphics[width=8in]{sofobomo2011-16}

{The Becci--Pergolesi Family Mausoleum Gate}\\
{\footnotesize Cancello della cappella della famiglia Becci--Pergolesi}
\vspace*{\fill}
\newpage

\includegraphics[width=8in]{sofobomo2011-17}

{The Tacchi Family Mausoleum Gate}\\
{\footnotesize Cancello della cappella della famiglia Tacchi}
\vspace*{\fill}
\newpage

\includegraphics[width=8in]{sofobomo2011-18}

{Mausoleum without a Name}\\
{\footnotesize Cappella senza nome}
\vspace*{\fill}
\newpage

\pagestyle{empty}
\vspace*{2in}

\hfill{\textsf{\Huge Flowers and Plants}}

\hfill{\textsf{\LARGE Fiori e piante}}

\clearpage{\pagestyle{empty}\cleardoublepage}

\pagestyle{plain}

\includegraphics[width=8in]{sofobomo2011-19}

{Tinted Bush}\\
{\footnotesize Cespuglio colorato}
\vspace*{\fill}
\newpage

\includegraphics[width=8in]{sofobomo2011-20}

{Parched Flowers}\\
{\footnotesize Fiori secchi}
\vspace*{\fill}
\newpage

\includegraphics[width=8in]{sofobomo2011-21}

{Chained Flowers}\\
{\footnotesize Fiori incatenati}
\vspace*{\fill}
\newpage

\includegraphics[width=8in]{sofobomo2011-22}

{Green Flowers}\\
{\footnotesize Fiori verdi}
\vspace*{\fill}
\newpage

\includegraphics[width=8in]{sofobomo2011-23}

{Fallen Flowers}\\
{\footnotesize Fiori caduti}
\vspace*{\fill}
\newpage

\includegraphics[width=8in]{sofobomo2011-24}

{Flowers in the Glass Pot}\\
{\footnotesize Fiori nel vaso di vetro}
\vspace*{\fill}
\newpage

\includegraphics[width=8in]{sofobomo2011-25}

{Grew on Rocks}\\
{\footnotesize Cresciuti sulla roccia}
\vspace*{\fill}
\newpage

\includegraphics[width=8in]{sofobomo2011-26}

{Upright}\\
{\footnotesize Eretto}
\vspace*{\fill}
\newpage

\pagestyle{empty}
\vspace*{2in}

\hfill{\textsf{\Huge Decorations}}

\hfill{\textsf{\LARGE Decorazioni}}

\clearpage{\pagestyle{empty}\cleardoublepage}

\pagestyle{plain}

\includegraphics[width=8in]{sofobomo2011-27}

{Count Pietro Benedetti Grave, Detail}\\
{\footnotesize Sepolcro del conte Pietro Benedetti, dettaglio}
\vspace*{\fill}
\newpage

\includegraphics[width=8in]{sofobomo2011-28}

{Count Pietro Benedetti Grave, Detail}\\
{\footnotesize Sepolcro del conte Pietro Benedetti, dettaglio}
\vspace*{\fill}
\newpage

\includegraphics[width=8in]{sofobomo2011-29}

{The Angeloni--Sollazzi Family Grave, Detail}\\
{\footnotesize Deposito della famiglia Angeloni--Sollazzi, dettaglio}
\vspace*{\fill}
\newpage

\includegraphics[width=8in]{sofobomo2011-30}

{The Fantini Family Grave, Detail}\\
{\footnotesize Deposito della famiglia Fantini, dettaglio}
\vspace*{\fill}
\newpage

\includegraphics[width=8in]{sofobomo2011-31}

{The Fantini Family Grave, Detail}\\
{\footnotesize Deposito della famiglia Fantini, dettaglio}
\vspace*{\fill}
\newpage

\includegraphics[width=8in]{sofobomo2011-32}

{Marietta Sbarbati Grave, Detail}\\
{\footnotesize Tomba di Marietta Sbarbati, dettaglio}
\vspace*{\fill}
\newpage

\includegraphics[width=8in]{sofobomo2011-33}

{The Francesco Canducci Family Grave, Detail}\\
{\footnotesize Deposito della famiglia Francesco Canducci}
\vspace*{\fill}
\newpage

\includegraphics[width=8in]{sofobomo2011-34}

{The Benedetti Forastieri Family Grave, Detail}\\
{\footnotesize Tomba della famiglia Benedetti Forastieri, dettaglio}
\vspace*{\fill}
\newpage

\pagestyle{empty}
\vspace*{2in}

\hfill{\textsf{\Huge Looking Outside}}

\hfill{\textsf{\LARGE Guardando all'esterno}}

\clearpage{\pagestyle{empty}\cleardoublepage}

\pagestyle{plain}

\includegraphics[width=8in]{sofobomo2011-35}

{The Hills around Senigallia}\\
{\footnotesize Le colline attorno Senigallia}
\vspace*{\fill}
\newpage

\includegraphics[width=8in]{sofobomo2011-36}

{The Town of Senigallia and the Seaside}\\
{\footnotesize La città di Senigallia ed il mare}
\vspace*{\fill}
\newpage

% End centering for the caption
\end{center}

\pagecolor{MyGray}
\pagestyle{empty}
\vspace*{2in}

\hfill{\textsf{\textcolor{white}{\Huge Colophon}}}

\hfill{\textsf{\textcolor{white}{\LARGE Colofone}}}

\clearpage{\pagestyle{empty}\cleardoublepage}

\pagecolor{white}
\pagestyle{plain}

\begin{Parallel}{0.47\textwidth}{0.47\textwidth}
\ParallelLText{
    \begin{english}[variant=american]
        {\textsf{{\LARGE About this book}}}
    \end{english}
}
\ParallelRText{
    \begin{italian}
        {\textsf{{\LARGE A proposito di questo libro}}}
    \end{italian}
}
\ParallelPar
\ParallelLText{
    \begin{english}[variant=american]
        Solo Photo Book Month (\href{http://sofobomo.org}{\tt sofobomo.org}) 
        is a DIY project to make a photo book in one month.
        All of the photos must be taken and the book laid out and processed 
        into a PDF in 31 days.
    \end{english}
}
\ParallelRText{
    \begin{italian}
        Solo Photo Book Month (\href{http://sofobomo.org}{\tt sofobomo.org}) 
        è un progetto fai da te per creare un libro fotografico in un mese.
        In 31 giorni bisogna scattare tutte le foto, progettare il libro
        e creare il PDF.
    \end{italian}
}
\ParallelPar
\ParallelLText{
    \begin{english}[variant=american]
        This is my first year participating in the event and I am glad I did it.
        I have been thinking about the Senigallia cemetery for some years now.
        I studied its history and the history of some of its monuments (see recommended reading).
        I thought about the best photographic approach and I took many many photos.
        But in the end, I didn't find the right ``inspiration'' and nothing was concluded.
    \end{english}
}
\ParallelRText{
    \begin{italian}
        Questo è il primo anno che partecipo all'evento e sono contento di averlo fatto.
        Era da qualche anno che pensavo al cimitero di Senigallia.
        Ho studiato la sua storia e la storia di alcuni dei suoi monumenti (vedi le letture consigliate).
        Ho pensato al migliore approccio fotografico e ho fatto molte molte foto.
        Ma alla fine, non ho trovato la giusta ``ispirazione'' e non ho concluso niente.
    \end{italian}
}
\ParallelPar
\ParallelLText{
    \begin{english}[variant=american]
        When I decided to participate at SoFoBoMo, I tackled those project again.
        This time, I didn't try to plan or to think in advance about the photos.
        I went to the Senigallia cemetery in an afternoon of a hot day of August.
        I wandered through the places that I already visited many times and
        I took photos of whatever I found interesting.
        In the following days, choosing and processing the photos,
        the themes of this book emerged naturally.
    \end{english}
}
\ParallelRText{
    \begin{italian}
        Quando ho deciso di partecipare a SoFoBoMo, ho affrontato ancora quel progetto.
        Questa volta, non ho cercato di pianificare o di pensare in anticipo alle foto.
        Sono andato al cimitero di Senigallia in un pomeriggio di un caldo giorno di Agosto.
        Ho camminato per i luoghi che avevo già visitato molte volte e
        ho fotografato quello che trovavo interessante.
        Nei giorni successivi, selezionando ed elaborando le foto, i temi di questo libro
        sono emersi naturalmente.
    \end{italian}
}
\ParallelPar
\ParallelLText{
    \begin{english}[variant=american]
        These themes gives only a partial view of the cemetery.
        Other interesting things were excluded.
        But I am pleased with the result, and I have finally created a book!
    \end{english}
}
\ParallelRText{
    \begin{italian}
        Questi temi danno solo una visione parziale del cimitero.
        Altre cose interessanti sono state escluse.
        Ma sono contento del risultato, e ho finalmente fatto un libro!
    \end{italian}
}
\ParallelPar
\ParallelLText{
    \begin{english}[variant=american]
        {\textsf{{\LARGE Technical details}}}
    \end{english}
}
\ParallelRText{
    \begin{italian}
        {\textsf{{\LARGE Dettagli tecnici}}}
    \end{italian}
}
\ParallelPar
\ParallelLText{
    \begin{english}[variant=american]
        Photos taken with a Nikon D50 and a Nikkor AF 35--70 mm 1:2.8 D lens.
        Selection and processing done with Adobe Lightroom.
    \end{english}
}
\ParallelRText{
    \begin{italian}
        Foto fatte con una Nikon D50 e un obiettivo Nikkor AF 35--70 mm 1:2.8 D.
        Selezione ed elaborazione fatta con Adobe Lightroom.
    \end{italian}
}
\ParallelPar
\ParallelLText{
    \begin{english}[variant=american]
        Book designed using \LaTeX\ and \XeTeX\ bundled with the Tex Live 2009 distribution packaged for Debian Linux.
        Many thanks to Eric Jeschke for his blog posts and \LaTeX\ examples. See \href{http://redskiesatnight.com}{\tt redskiesatnight.com}.
    \end{english}
}
\ParallelRText{
    \begin{italian}
        Libro progettato utilizzando \LaTeX\ e \XeTeX\ inclusi nella distribuzione Tex Live 2009 per Debian Linux.
        Molte grazie a Eric Jeschke per gli articoli nel suo blog e per i suoi esempi di \LaTeX. Vedi \href{http://redskiesatnight.com}{\tt redskiesatnight.com}.
    \end{italian}
}
\ParallelPar
\ParallelLText{
    \begin{english}[variant=american]
        Fonts used \href{http://www.exljbris.com/fontin.html}{\textbf{Fontin}} and \href{http://www.exljbris.com/fontinsans.html}{\textsf{\textbf{Fontin Sans}}} from exljbris font foundry, \href{http://www.lekton.info/}{\texttt{\textbf{Lekton}}} from ISIA Urbino.
    \end{english}
}
\ParallelRText{
    \begin{italian}
        Caratteri usati \href{http://www.exljbris.com/fontin.html}{\textbf{Fontin}} e \href{http://www.exljbris.com/fontinsans.html}{\textsf{\textbf{Fontin Sans}}} realizzati da exljbris font foundry, \href{http://www.lekton.info/}{\texttt{\textbf{Lekton}}} realizzato dall'ISIA di Urbino.
    \end{italian}
}
\ParallelPar
\ParallelLText{
    \begin{english}[variant=american]
        \LaTeX\ files put under a Creative Commons Attribution-Share Alike license
        and shared on github: \href{https://github.com/maurotrb/sofobomo2011}{\tt github.com/maurotrb/sofobomo2011}.
    \end{english}
}
\ParallelRText{
    \begin{italian}
        File \LaTeX\ rilasciati con licenza Creative Commons Attribution-Share Alike
        e condivisi su github: \href{https://github.com/maurotrb/sofobomo2011}{\tt github.com/maurotrb/sofobomo2011}.
    \end{italian}
}
\ParallelPar
\ParallelLText{
    \begin{english}[variant=american]
        {\textsf{{\LARGE Recommended Reading}}}
    \end{english}
}
\ParallelRText{
    \begin{italian}
        {\textsf{{\LARGE Letture consigliate}}}
    \end{italian}
}
\ParallelPar
\ParallelLText{
    \begin{english}[variant=american]
        There are two interesting works on the history of the cemetery and on its most important tombs.

        Unfortunately they are only in Italian. See on right column.
    \end{english}
}
\ParallelRText{
    \begin{italian}
        Laura Casavecchia,
        \emph{Un cimitero ``ballerino''. Senigallia, i Mastai e la questione del camposanto nel XIX secolo.}
        Tesi di Laurea in Storia della città e del territorio, Università degli Studi di Bologna.
        2006.
        Reperibile all'indirizzo \href{http://librisenzacarta.it/2007/05/01/un-cimitero-ballerino/}{\tt librisenzacarta.it/2007/05/01/un-cimitero-ballerino/}.
    \end{italian}
}
\ParallelPar
\ParallelLText{
    \begin{english}[variant=american]
    \end{english}
}
\ParallelRText{
    \begin{italian}
        Donato Mori,
        \emph{Visita Guidata ad alcune significative tombe ottocentesche nel Cimitero senigalliese.}
        Distribuzione gratuita.
        2007.
        Reperibile presso la Biblioteca Comunale Antonelliana di Senigallia con la collocazione
        \textsc{miscellanea senigalliese} 18152.
    \end{italian}
}
\ParallelPar
\end{Parallel}

\vspace*{\fill}

\begin{center}
{\textsf{{\large For comments and questions --- Per commenti e domande}}}

{\tt \href{mailto:MauroTaraborelliPhoto@gmail.com}{MauroTaraborelliPhoto@gmail.com}}

{\tt \href{http://www.maurotaraborelliphoto.com}{www.maurotaraborelliphoto.com}}

\begin{pspicture}(1in,1in)
\psbarcode{http://www.maurotaraborelliphoto.com}{}{qrcode}
\end{pspicture}

\end{center}

%END


% close the document environment
\end{document}
